Determining the cultural values of a given population is of interest to various institutions and fields of research. The World Values Survey (WVS) [https://www.worldvaluessurvey.org/] is perhaps the most famous example of such an endeavour. Conducting and releasing national surveys circa twice a decade since 1981, the WVS is exploring how values have changed over time in the more than 90 countries in which the survey has been conducted. Viewed in the context general opinion polling, the WVS is relatively new: According to [https://doi.org/10.1177/107769907204900219] the earliest example of modern public opinion polling, tallying voter preferences, was conducted by the Raleigh Star and the North Carolina State Gazette for the 1824 American presidential election. Public opinion, as a concept, though, has been known since at least the time of the Roman Empire [https://doi.org/10.1017/9781316535158].

Notions of what exactly modern surveys of public opinion measure—even when narrowing the focus to cultural values—varies and is under constant development. Hofstede's cultural dimensions theory  through factor analysis concludes that culture can meaningfully be described as being six dimensional. Shalom H. Schwartz's Theory of Basic Human Values , building on Hofstede's work, concludes that 10 is a more appropriate dimension count. An extended Schwartz's theory includes 19 individual values. Jonathan Haidt's Moral Foundations Theory cover similar ground with its five (sometimes six) dimensions.

A clash of levels of abstraction must here be noted: should cultural values should be thought of as a distributional feature of the set of values of the individuals in a culture, or could cultural values meaningfully diverge from the value of individuals inside it? From a functionalist perspective this is possible, as power is not equally distributed between individuals within culture. Game theoretically there could even be a complete misalignment between cultural values and individual values, though in free societies this is largely a theoretical possibility (CITE).

Inglehart-Welzel cultural map of the world is derived from the union of WVS and the European Social Survey surveys. The ESS is one obvious example of such: with the first round released in 2002 10 bi-yearly surveys has now been conducted in more than 30 European countries (CITE ESS). Common among all surveys is that they are expensive; an ESS round consists of about 2,000 in person interviews in each of the usually more than 20 countries in focus (40000); time consuming, and much more detailed than would be needed to answer whatever question about culture we are attempting to answer. We often do not care about the individuals, but rather the group. There are also privacy concerns, due to the sensitive nature of many questions asked. Being able to infer the cultural vales we are surveying about, from freely available data, could yield improvements in cost: actual in person interviews could perhaps be conducted with lower frequency; and speed: if the data from which we are inferring is available, say daily, as it is with Wikipedia activity, daily, as opposed to biannual insight, could be gained.

Quantifying the ways in which cultures differ from one another has been done numerous times with various different intentions and levels of abstraction. In various ways. The Google Ngram Viewer quantitative analysis analysis of digitised texts
